\begin{resumo}
Os recursos tecnológicos estão cada vez mais presentes no âmbito escolar, por ser muito importante para a construção de pensamento crítico, social e humano, no entanto, utilizá-los na escola é um grande desafio para muitos professores. O objetivo desse trabalho é o desenvolvimento de uma ferramenta educacional que auxilie o professor no uso de vídeos na Educação a Distância, permitindo a construção de cursos com questionários associados à vídeos integrados aos sistemas de gestão de aprendizado como o \textit{moodle}, através da \textit{Learning Tool Interoperability}. Para alcançar esse objetivo, foram usadas as tecnologias de desenvolvimento \textit{PHP} suportada pelo \textit{framework laravel} e \textit{Javascript} com a bibliotecas \textit{Vue.JS e jQuery}. Também foi usada uma biblioteca \textit{LTI} desenvolvida em \textit{PHP}, que apesar de ajudar na integração, precisou ser atualizada para atender os requisitos do desenvolvimento.

 \vspace{\onelineskip}
    
 \noindent
 \textbf{Palavras-chaves}: Educação a Distância, Vídeos, Sistema de gestão de aprendizado \textit{Learning Tool Interoperability}
\end{resumo}
