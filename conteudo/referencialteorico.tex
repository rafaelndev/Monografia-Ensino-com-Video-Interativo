\chapter[Referencial Teórico]{Referencial Teórico}

\section{USO DAS TECNOLOGIAS NA EDUCAÇÃO}

As tecnologias chegaram as escolhas, segundo \citeonline{moran}, apesar da resistência institucional, as mudanças são necessárias. As empresas estão muito ativas na educação online e buscam agilidade das universidades, flexibilização e rapidez na educação continuada. Os avanços na educação à distância com a Internet é bem notável. A Internet retirou a isolação da educação a distância, de atraso ou de ensino de segunda classe. A interconectividade que a Internet desenvolveu nesses últimos anos, começou a revolucionar a forma de ensinar e aprender.
\par
Em um artigo publicado na revista Veja, chamado "O computador não educa ensina" por \citeonline{veja-educacao}, pergunta como as escolas irão fazer uso do computador um instrumento para mudar a velha escola, praticamente congelada no tempo desde o século XIX? A publicação mostra experiência de países que utilizam essa ferramenta no processo de ensino aprendizagem, como é o caso do Japão, o autor diz: "Estudar em rede lá se tornou uma febre". No Japão as escolas estão ensinando em rede, e isso abre uma nova dimensão no exercício intelectual, em que as crianças são incentivadas a desenvolver com rapidez de raciocínio para dar respostas on-line e a expor ideias para centenas de colegas virtuais, ajudando também no quesito de trabalho em equipe do aluno.
\par
Para \citeonline{goncalves}  a tecnologia é muito mais do que apenas equipamentos, máquinas e computadores. A organização só funciona a partir da operação de dois sistemas que dependem um do outro, o sistema técnico, formado pelas técnicas e ferramentas usadas para realizar cada tarefa, e o sistema social, com suas necessidades, expectativas, e sentimentos sobre o trabalho. Os dois sistemas são usados de forma otimizada quando os requisitos da tecnologia e as necessidades das pessoas são atendidos conjuntamente. Dessa forma é possível distinguir entre tecnologia(conhecimento) e sistema técnico (combinação de máquinas e métodos para obter um resultado desejado).
\par
Seguindo o pensamento de Gonçalves, um artigo da Gazeta do Povo por \citeonline{juliana-gazeta}, afirma que "O computador não ensina nada sozinho", assim como a presença de ou não de um laboratório de informática, não quer dizer muita coisa sobre a qualidade de ensino de uma escola, mais que um bom planejamento faz a diferença, complementa também que pouco adianta usar o computador em sala seguindo uma metodologia tradicional, como uma ferramenta para fazer apenas cópias de texto, pois isso pode ser feito com lápis e caderno.

\section{EDUCAÇÃO À DISTÂNCIA (EAD)}

\citeonline{moran-ead} descreve a Educação a Distância como um processo de ensino-aprendizagem, mediado por tecnologias, onde professores e alunos estão separados espacial e/ou temporalmente.
\par
O conceito de distância em EaD deve ser entendida como uma separação espacial (geográfica/local) entre participantes do processo educacional, sejam alunos ou professores. Quando o estudo ocorre pela internet, alunos e professores podem estar em lugares diferentes, e ainda assim possam acessar o curso e os materiais e recursos educativos em momentos distintos.
\par
\citeonline[p.~3]{vilacca} salienta que as formas de distância podem gerar incompreensões, criando preconceitos em relação a EaD. A distância não implica necessariamente em divergência temporal (cronológica), alunos e professores podem estar em locais diferentes, participando de forma síncrona de uma mesma atividade, como por \textit{chat}. \citeonline[p.~19]{valente}  destacam também que o distanciamento físico entre os participantes "não implica em distanciamento humano", os autores descrevem também que a EaD, "possibilita a manipulação do espaço e do tempo em favor da educação." \Idem[p.~20]{valente}.
\par
O conceito de curso e de aula também é um pouco diferente do que entendemos hoje, de uma aula com espaço e tempo determinados, com a internet por exemplo, esse tempo e espaço fica cada vez mais flexível. Esse processo é descrito por \citeonline[p.~2]{moran-ead} em seu artigo:
\begin{citacao}
  "O professor continuará "dando aula", e enriquecerá esse processo com as possibilidades que as tecnologias interativas proporcionam: para receber e responder mensagens dos alunos, criar listas de discussão e alimentar continuamente os debates e pesquisas com textos, páginas da internet, até mesmo foram do horário específico da aula."
\end{citacao}
\par
Dessa forma o Ensino a Distância também pode ser usado como um complemento a uma aula presencial tradicional, enriquecendo-a.
\par
O EaD é uma forma do aluno aprender a distância, por vários meios, como correspondências, televisão, radio e internet. Com a união do Ensino a distância com a \textit{Internet}, surgiu um novo termo conhecido como \textit{e-Learning}, uma modalidade de aprendizagem baseada na Internet.

\newpage
\subsection{\textit{ELETRONIC LEARNING (E-LEARNING)}}
Segundo \citeonline{rodrigues} a \textit{e-Learning} é uma modalidade de ensino a distância que possibilita a auto aprendizagem, com recursos didáticos sistematicamente organizados, apresentados em diferentes suportes tecnológicos de informação, utilizados de forma isolada ou combinada através da internet.
\par
\citeonline{Felipini} acrescenta que a \textit{e-Learning} é basicamente um sistema em um servidor, que transmite através da Internet ou Intranet, informações e instruções aos alunos para agregar um conhecimento especifico. As etapas de ensino no \textit{e-Learning} são pré-programadas, podem ou não ser divididas em módulos e para o ensino, podem ser usados diversos recursos como e-mail, textos, imagens, texto, \textit{chat}, \textit{links} de informação externa, vídeos e teleconferências.
\par
O \textit{e-learning} também pode ser definido como um método de ensino a distância que usa novas tecnologias multimédia e Internet para promover a qualidade da formação, facilitando assim o acesso a recursos e serviços, assim como trocas de informações entre os diversos intervenientes envolvidos. \cite{spi}.
\par
\cite[p.~3]{barbosa} descreve as necessidades de um sistema de \textit{e-Learning}:
\begin{citacao}
  "Para que seja possível uma aula/formação, com o modelo de e-Learning, é necessário que o e-tutor disponibilize os e-conteúdos (ou recursos didácticos dos cursos), bem como as interacções com os e-alunos. Para isso, há o acesso à plataforma, através de uma password cedida pelo formador ou entidade. Através dessa password, o e-aluno, tem acesso a toda informação, que inclui normalmente textos, imagens e animações."
\end{citacao}
\Idem{barbosa} complementa o pensamento com uma imagem que mostra o processo do \textit{e-Learning}.

\begin{figure}[h]
  \centering
  \label{fig:e-learning-caracterização-barbosa}
  \includegraphics[keepaspectratio=true,scale=0.6]{figuras/e-learning-barbosa.png}
  \caption{Processo de \textit{e-Learning} por \citeonline{barbosa}}
\end{figure}

\par
É importante deixar claro que existem diferenças entre o \textit{e-Learning} e a EAD, enquanto a EAD é definido como qualquer ensino a distância, ou seja uma forma de educação em que o professor, conteúdo e aluno podem estar em locais e tempos diferentes, o \textit{e-Learning} é uma forma de Educação a distância que utiliza os meios tecnológicos como a Internet.


\subsection{AS VANTAGENS E DESVANTAGENS DO \textit{E-LEARNING}}
As principais vantagens do \textit{e-learning}, mostradas por \citeonline[p.~80]{hall} em seu artigo são: multiplicação dos pontos de treinamento, facilidade de acesso das informações, diminuição dos custos, remoção dos deslocamentos de funcionários, minimização do tempo usado em treinamentos, rapidez na atualização de conteúdos e definição do ritmo de aprendizagem pelo próprio aluno.
\citeonline[p.~5]{barbosa} aponta a flexibilidade e a disponibilização da informação em tempo real como uma vantagem importante do \textit{e-Learning}, assim o aluno consegue ter um tempo próprio para aulas, podendo acessar as aulas a qualquer hora e local.
\par
Segundo \cite[p.~6]{bernardo} no \textit{e-learning}, o sistema computadorizado ou o treinador podem detectar mais facilmente falhas ou dificuldades dos alunos e corrigi-las imediatamente, de modo que isso não afete negativamente o aprendizado.
\par
No entanto a alta robotização do treinamento pode gerar dificuldades tanto em alunos como em professores, por ser algo relativamente novo, não existem muitas pessoas com experiência para desenvolver aulas on-line. \cite[p.~6]{bernardo} aponta para o problema de que os alunos podem não se sentir à vontade para participar de aulas virtuais, porque acham muito importante o encontro físico entre alunos e professores na sala de aula.
\par
A plataforma também deve ser simples de manusear, para facilitar a interatividade, tendo o apoio do formador, para poder tirar dúvidas, partilhar informações e experiências. Se os alunos tiverem dificuldades em usar a tecnologia disponibilizada na plataforma, podem ter que dedicar mais tempo para aprender a usar a plataforma, do que o acesso ao seu conteúdo. \apud[p.~5]{barbosa}{choi}
\par
\Idem[p.~6]{barbosa} aponta também uma controvérsia da "ausência" de relação humana, o que pode gerar ou não isolamento por parte do aluno. Mesmo que o aluno possa estar sozinho no local de acesso, ele poderá ter a companhia dos colegas virtuais e até do professor.

\section{E-LEARNING APOIADO POR VÍDEOS}
O vídeo se tornou um recurso de fácil acesso, mais seu na educação somente começou a partir da década de 90, \citeonline{moran} foi um dos pioneiros a escrever sobre esse assunto no Brasil com o artigo "O Vídeo na Sala de Aula", no artigo o autor fala sobre as linguagens da TV e do vídeo, e também sobre seu impacto na comunicação.
\par
\citeonline[p.~3]{moran} destaca pontos importantes na utilização de vídeos na educação como: auxiliar o despertar da curiosidade, permite desenvolver cenários desconhecidos pelos alunos, proporciona simulações de realidade, reproduz também documentários, entrevistas, depoimentos e ajuda no desenvolvimento do senso crítico. Segundo \citeonline[p.~3]{moran}:
\begin{citacao}
  As tecnologias são pontes que abrem a sala de aula para o mundo, que representam, medeiam o nosso conhecimento do mundo. São diferentes formas de representação da realidade, de forma mais abstrata ou concreta, mais estática ou dinâmica, mais linear ou paralela, mas todas elas, combinadas, integradas, possibilitam uma melhor apreensão da realidade e o desenvolvimento de todas as potencialidades do educando, dos diferentes tipos de inteligência, habilidades de atitudes.
\end{citacao}

\section{SISTEMA DE GESTÃO DE APRENDIZADO (\textit{LMS})}

\citeonline[p.~2]{andrade} descreve o \textit{LMS} como um software que controla o desenvolvimento, gerenciamento e acompanhamento de cursos de aprendizagem online. O \textit{LMS} é um sistema de gestão que possui funcionalidades para suportar o aprendizado a distância tais como: distribuição, acompanhamento, monitoramento e administração de conteúdo de aprendizagem com os progressos e interações dos alunos.
\par
Segundo \citeonline[p.~3]{goni}:
\begin{citacao}
  Um \textit{LMS} tem como um dos objetivos, simplificar a administração dos programas de treinamento e ensino em uma organização. O sistema auxilia no planejamento dos processos de aprendizagem e ainda permite que os participantes colaborem entre si através da troca de informações e conhecimentos.
\end{citacao}
\par
Esses sistemas ajudam na disponibilidade das informações, análises, rastreamento de dados e a geração de relatórios sobre o andamento dos cursos.

As principais funcionalidades do sistema \textit{LMS}, são:
\begin{itemize}
  \item Criar e administrar cursos;
  \item Oferecer ferramentas de comunicação como listas de discussão, \textit{chats} e mensagens instantâneas;
  \item Administrar grades curriculares e listagens de espera;
  \item Fornecer tarefas, avaliações e exercícios;
  \item Monitorar o acesso do usuário;
  \item Administrar matrículas de aprendizes;
  \item Gerar relatórios e informações sobre o desempenho dos aprendizes, etc;
\end{itemize}