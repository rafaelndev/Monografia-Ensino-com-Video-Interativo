\chapter{Introdução}
%\addcontentsline{toc}{chapter}{Introdução}
% * <leandro@eunapolis.ifba.edu.br> 2015-09-17T22:16:10.530Z:
%
%  É necessário? addcontentsline{toc}{chapter}{Introdução}?
%
Os meios de comunicação evoluíram rapidamente em pouco tempo, possibilitando entrar em contato com pessoas de todo mundo em tempo real. Sem os recentes avanços teríamos de esperar dias ou até meses para transmitir informações à pessoas que se encontram em lugares distantes.
% * <leandro@eunapolis.ifba.edu.br> 2015-09-17T22:17:43.656Z:
%
%  Reescrever paragrafo, "comunicação... comunicar...comunicação..." torna a leitura repetitiva
%
% ^ <rafaeln.dev@gmail.com> 2015-10-10T22:44:18.799Z:
%
%  Dei uma melhorada.
%

% * <leandro@eunapolis.ifba.edu.br> 2015-09-17T22:17:53.171Z:
%
%  Retirar par
%
% ^ <rafaeln.dev@gmail.com> 2015-10-10T22:44:36.720Z:
%
%  acho que removi todos agora
%
%
A era da Informação reduz os espaços e aproxima as pessoas de todo mundo, trazendo oportunidades na área de treinamento e ensino. Com isso, novos recursos podem ser usados na educação a distância, como o desenvolvimento da modalidade \textit{e-learning} ou educação \textit{on-line}

Sendo assim, foram criados ambientes para a melhoria da educação à distância que permitem aos alunos cursarem disciplinas, ou até cursos inteiros, sem sair de casa e sem a presença física de um professor. Para isso, necessitando apenas de um computador com acesso a internet e um navegador (browser).

Essa plataforma, é chamada de Ambiente Virtual de Aprendizagem (AVA), que funciona como uma sala de aula virtual. Por ela o professor possui funcionalidades que podem ajudar na avaliação do aluno por meio de um curso disponibilizado nesse ambiente. Também é possível inserir vários recursos audiovisuais aos cursos e realizar integrações com outros sistemas.

O vídeo tem sido uma das tecnologias mais populares usadas como recurso pedagógico, auxiliando o professor. O motivo dessa popularidade se deve ao fato de que muitos alunos aprendem melhor com estímulos audiovisuais, em comparação à uma educação tradicional.

Segundo \citeonline[p.~3]{vicentini} a popularização da Internet em conjunto com o custo reduzido de filmadoras e máquinas digitais concederam às pessoas a possibilidade de produzir e distribuir seu próprio material audiovisual.

\citeonline{moran}, descreve o vídeo como um recurso que está sempre ligado a um contexto de lazer e entretenimento, fazendo que com os alunos pensem que o vídeo em sala de aula significa descanso e não uma aula. Por isso é preciso aproveitar esse olhar do aluno diante do vídeo, e estudar maneiras para atrair o aluno para os assuntos do planejamento pedagógico, e estabelecer pontes entre o vídeo e outras dinâmicas da aula.

Quando o vídeo é usado como conteúdo de ensino, ele se torna em uma ferramenta que faz com que os estudantes tenham interesse em explorar o assunto abordado com mais profundidade, incentivando-o a buscar por mais conteúdo sobre o assunto. Segundo \cite{moran}, o vídeo é sensorial, que seduz, informa, entretém e projeta outras realidades no imaginário, combinando a comunicação sensorial com a audiovisual, consegue trabalhar com o emocional e racional do indivíduo, por isso o uso de vídeos, principalmente na educação, não deve ser negligenciado por ter essa enorme capacidade de sensibilização e motivação dos alunos.

O uso eficiente de vídeos agregado a pedagogia e integrado a temas trabalhados pelo professor, torna o aprendizado mais significativo. Segundo~\cite{almeida} na integração de ensino com a tecnologia, é preciso levantar problemáticas relacionadas ao mundo e ao contexto e a realidade do aluno, com questões temáticas em estudo.

Dentre as ferramentas já existentes que permitem o uso de vídeos de forma mais interativa temos o \textit{Youtube}\footnote{\url{http://youtube.com}} que possui uma ferramenta de edição que permite adicionar notas ao vídeo, mas não permite a interação por meio de perguntas. Existe também o \textit{blubbr}\footnote{\url{http://blubbr.com}} que permite adicionar questões comuns à um vídeo com tempo predefinido, mas não possui nenhuma forma de integração com os \ac{SGA}. E por fim, o \textit{Huzzaz}\footnote{\url{http://huzzaz.com}} que permite a adição de questões aos vídeos, também fornece notas e informações sobre a performance dos alunos em um painel acessado pelo professor.

Pretende-se com essa pesquisa, o desenvolvimento de uma ferramenta de suporte ao aprendizado, visando o uso otimizado de vídeos no ensino-aprendizagem. Fazendo com que os vídeos se tornem uma ferramenta para auxiliar o ensino com a interatividade do vídeo e integração com questões desenvolvidas pelo professor, dessa forma o aluno poderá assistir aos vídeos no ambiente do sistema, e ao mesmo tempo responder questionários referentes ao vídeo e/ou com ligação ao ensino passado pelo professor.

\section{Objetivos}
\subsection{Objetivos Gerais}
A partir das pesquisas realizadas, foi possível perceber a escassez de ferramentas que auxiliem o professor com o uso de vídeos no ensino a distância.

O trabalho tem como objetivo geral desenvolver uma ferramenta que permita ao professor construir cursos com questionários associados a vídeos e integrados aos sistemas de gestão de aprendizado.
% ----------- FIM OBJETIVOS GERAIS -----------
\subsection{Objetivos Específicos}
\begin{itemize}
  \item Desenvolvimento de uma aplicação que possibilite o professor utilizar vídeos, em uma interface interativa com questionários.
  \item Integrar a ferramenta a os sistemas de gestão de aprendizagem.
  \item Permitir que o aluno possa fazer o curso e receber a nota diretamente no \ac{SGA}.
\end{itemize}
% ----------- FIM OBJETIVOS ESPECÍFICOS -----------
\section{Organização}

Esse trabalho está distribuído na seguinte forma: no Capitulo 1 foi feita uma breve introdução, apresentando a justificativas e objetivos do trabalho. O Capitulo 2 apresenta a o referencial teórico do trabalho, descrevendo os conceitos necessários para o entendimento do do texto de acordo com os temas apresentados no projeto. Já Capitulo 3 apresenta uma descrição geral das tecnologias utilizadas no projeto, explicando também o motivo da escolha das mesmas. No Capitulo 4 é apresentado como foi feito a implementação do projeto, e como as tecnologias foram utilizadas para alcançar os objetivos. Por fim, no capitulo 5 apresenta a conclusão do projeto, a partir dos resultados obtidos. Também são feitas propostas de trabalhos futuros baseando-se no projeto.
% ----------- FIM ORGANIZAÇÃO -----------
