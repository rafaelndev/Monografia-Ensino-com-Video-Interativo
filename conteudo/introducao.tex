\chapter{Introdução}
\addcontentsline{toc}{chapter}{Introdução}
Os meios de comunicação evoluíram rapidamente em pouco tempo, permitindo que possamos nos comunicar com pessoas em todo mundo em tempo real, sem os meios de comunicação modernos teríamos de esperar meses ou anos para transmitir informações para pessoas que se encontram em lugares distantes, e atualmente podemos enviar e receber informações em segundos.
\par
A era da Informação reduziu os espaços e aproxima as pessoas de todo mundo, trazendo também oportunidades em outros campos, como a área de treinamento e ensino, possibilitando novas formas de interação entre as pessoas, que viabilizou novos recursos que podem ser usados na educação a distância, como por exemplo o desenvolvimento da modalidade \textit{e-learning} ou educação \textit{on-line}
\par
Com isso, foram criados ambientes para a melhoria da educação a distância, juntamente com as tecnologias associadas à internet, fornece um ambiente com recursos que podem ser usados no ambiente acadêmico, permitindo também que o aluno possa cursar disciplinas ou cursos inteiros sem sair de casa, sem a presença física de um educador, necessita somente de um computador com acesso a internet, e um navegador (\textit{browser}).
\par
Essa plataforma, é chamada de Ambiente Virtual de Aprendizagem (AVA), que funciona como uma sala de aula virtual, por ela o professor possui funcionalidades que podem ajudar na avaliação do aluno por meio de um curso disponibilizado nesse ambiente. Também é possível inserir vários recursos nos cursos, como audiovisuais e realizar integrações com outros sistemas.
\par
Uma das várias tecnologias usadas em conjunto ao à educação \textit{on-line} e os AVA, o vídeo tem sido uma das mais populares, pois segundo \citeonline[p.~3]{vicentini} a popularização da Internet em conjunto com o custo reduzido de filmadoras e máquinas digitais concederam às pessoas e possibilidade de produzir e distribuir seu próprio material audiovisual.
\par
Os vídeos tem sido cada vez mais usados como recurso pedagógico. O vídeo ajuda o professor na tarefa de ensino, atraindo os alunos à um ensino que respeita as ideias de múltiplos estilos de aprendizagem e de múltiplas inteligências. Muitos alunos aprendem melhor com estímulos audiovisuais, em comparação a uma educação tradicional.
\par
\begin{citacao}
  \label{cit: moran}
  "O vídeo está umbilicalmente ligado à televisão e a um contexto de lazer, e entretenimento, que passa imperceptivelmente para a sala de aula. Vídeo, na cabeça dos alunos, significa descanso e não "aula", o que modifica a postura, as expectativas em relação ao seu uso. Precisamos aproveitar essa expectativa positiva para atrair o aluno para os assuntos do nosso planejamento pedagógico. Mas ao mesmo tempo, saber que necessitamos prestar atenção para estabelecer novas pontes entre o vídeo e as outras dinâmicas da aula." \cite{moran}
\end{citacao}
\par
Quando o vídeo é usado como conteúdo de ensino, se torna em uma ferramenta que faz com que o estudantes tenham interesse em explorar o assunto abordado com mais profundidade, incentivando-o a buscar por mais conteúdo sobre o assunto. Segundo \cite{moran}, o vídeo é sensorial, que seduz, informa, entretém e projeta outras realidades no imaginário, combinando a comunicação sensorial com a audiovisual, consegue trabalhar com o emocional e racional do individuo, por isso o uso de vídeos, principalmente na educação, não deve ser negligenciado, por ter essa enorme capacidade de sensibilização e motivação dos alunos.
\par
O vídeo e visto como um ótimo recurso para mobilizar os alunos, com problemáticas para despertar-lhes o interesse para realizar estudos em determinados temas, ou até trazer novas perspectivas para investigações em andamento.
\par
O uso eficiente de vídeos agregado a pedagogia integrado a temas trabalhados pelo educador, torna o aprendizado mais significativo. Segundo \cite{almeida} na integração de ensino com a tecnologia, é preciso levantar problemáticas relacionadas ao mundo e ao contexto e a realidade do aluno, com questões temáticas em estudo. 
\par
Pretende-se com essa proposta de pesquisa, o desenvolvimento de uma ferramenta de suporte ao aprendizado, visando o uso otimizado de vídeos no ensino-aprendizagem.
\par
Considera-se que a realização desse trabalho, seja importante para o âmbito acadêmico, pois, a aplicação a ser desenvolvida, almeja a solução do problema citado na Introdução [\ref{cit: moran}], fazendo com que os vídeos se tornem uma ferramenta para auxiliar o ensino com a interatividade do vídeo e integração com questões desenvolvidas pelo educador, dessa forma o aluno poderá assistir aos vídeos no ambiente do sistema, é ao mesmo tempo responder questionários referentes ao vídeo e/ou com ligação ao ensino passado pelo professor.

\section{Motivação}
% ----------- FIM MOTIVAÇÃO -----------
\section{Objetivos}
\subsection{Objetivos Gerais}
\begin{itemize}
  \item Desenvolver uma aplicação que sirva como plataforma de integração de questionários e vídeos interativos.
\end{itemize}
% ----------- FIM OBJETIVOS GERAIS -----------
\subsection{Objetivos Específicos}
\begin{itemize}
  \item Estudo e desenvolvimento de uma aplicação que possibilite o professor utilizar vídeos, em uma interface interativa de questionários e discussões.
\end{itemize}
% ----------- FIM OBJETIVOS ESPECÍFICOS -----------
\section{Organização}
% ----------- FIM ORGANIZAÇÃO -----------